\documentclass[10pt,a4paper]{article}
\usepackage[utf8]{inputenc}
\usepackage{amsmath}
\usepackage{amsfonts}
\usepackage{amssymb}
\usepackage{geometry}
\usepackage{verbatim}
\usepackage{enumerate}
\usepackage{fancyvrb}
\usepackage{graphicx}
\usepackage{hyperref}
\usepackage{tikz}
\usetikzlibrary{positioning}
\usetikzlibrary{shapes,snakes}
\usepackage[english]{babel}

\geometry{legalpaper, margin=1.5in}

\author{William Schultz}
\begin{document}
\title{Notes on Theory of Computation}
\author{William Schultz}
\maketitle

\section{Decidability}

A language $L$ is a set of strings $w \in \Sigma^*$, where $\Sigma$ is some finite alphabet i.e. some finite set of symbols. We say that a language is \textit{decidable} if there exists a Turing machine $M$ such that $M$ accepts every input $w \in L$ and always halts. We say that a language is \textit{recognizable} if there is a Turing machine $M$ that accepts every input $w \in L$ but it may reject or loop forever on inputs $w \notin L$. Note that for any Turing machine on a given input, there are three possible outcomes: accept, reject, or loop forever. A basic undecidable problem is the $A_{TM}$ problem, which asks, given a Turing machine $M$ and input $w$, does machine $M$ accept on $w$? Note that the class of recognizable languages is more general than the class of undecidable languages. That is, any language that is decidable is recognizable, but the converse is not necessarily true.

\section{P and NP}

The class $P$ is the set of languages for which there exists a Turing machine (i.e. an algorithm) that can decide $P$ in polynomial time $O(n^k)$, for some positive constant $k$. The class $NP$ is defined as the set of languages for which there exists a polynomial time \textit{verifier}. Formally, a language $L$ is in $NP$ if there exists a Turing machine $V$ (a verifier) such that, for any input $w \in L$ 
there exists a certificate $c$ such that $V$ accepts $\langle w,c \rangle$ and runs in polynomial time. Note that the size of the certificate can only be polynomial in the size of the input $w$, since the verifier machine only has a polynomial time run-time budget. At a high level, $NP$ is the class of languages for which there exists an efficient way to verify solutions to a given problem instance. There may not be an efficient algorithm to decide the answer to a given instance though. It is not known for certain, however, whether there are (or aren't) polynomial time algorithms for solving problems in $NP$. This is the famous $P$ vs. $NP$ problem. It is believed that problems in $NP$ do not have efficient (polynomial time) algorithms, but this is not formally proven. 

\subsection{NP-Completeness}

There are some problems in the class $NP$ that are the ``hardest'' problems in $NP$. We call these problems $NP$-complete. Formally, we say that a language $A$ is $NP$-complete if $A \in NP$ and every language $B \in NP$ is polynomial time reducible to $A$. For two languages $A$ and $B$, we say that $A$ is polynomial time reducible to $B$ if there is a polynomial time converter $R_{A\rightarrow B}$ that converts an input $w_A$ to an input $w_B = R_{A\rightarrow B}(w_A)$ so that $w_B \in B \iff w_A \in A$. So, if a problem is $NP$-complete, it means that every problem in $NP$ can be reduced to it i.e. we can take an input of any problem in $NP$, convert it to an input for the NP-complete problem in a way that preserves correctness. So, this means that if we solve one $NP$ complete problem in an efficient (polynomial time) algorithm, then all $NP$ problems are efficiently solvable. 

The canonical $NP$-complete problem is the SAT problem i.e. checking whether a boolean formula is satisfiable. The Cook-Levin theorem shows that SAT is $NP$-complete. With this knowledge, we can prove other problems $NP$-complete. Since we know that any $B \in NP$ can be reduced to an $NP$-complete problem $A$, we can show that $B$ is $NP$-complete by showing that $A$ is reducible to $B$. That is, we establish that $A$ is reducible to $B$ and $B$ is reducible to $A$, so the problems are ``equivalently" hard. Note that there are some problems in $NP$, however, that are not $NP$-complete. Ladner's theorem establishes this i.e. it proves the existence of problems that are in $NP$ but not in $P$ and are not $NP$-complete. This class is called $NP$-intermediate. The construction used in this theorem is complicated, though, and not necessarily ``natural''. There are problems that are suspected to be in $NP$-intermediate i.e. they are in $NP$ but have not been shown to be in $P$ or be $NP$-complete e.g. integer factorization.

\section{More Random Notes}

\subsection{Upper and Lower Bounds}
For any decision problem, we can establish both \textit{upper bounds} and \textit{lower bounds} on its complexity. Recall that a decision problem is formulated in terms of a language $L$, consisting of a set of strings. The decision problem for a given language $L$ is to determine whether $w \in L$ for some given string $w$.

An upper bound makes a statement about the maximum hardness/complexity of the problem, and a lower bound makes a statement about the minimum easiness of the problem. Establishing an upper bound is typically much easier, since you only need to provide a concrete algorithm that solves the problem in some worst case running time (i.e. show there exists an algorithm). Establishing a lower bound is generally much harder, since you need to show that there exists no algorithm that can solve the problem more efficiently than a certain complexity class (i.e. show no algorithm exists). 

For example, one of the best known algorithms for 3-SAT as of 2019 has a numerical upper bound of something around $O(1.307^n)$ \cite{2019fasterkSAT}. There exists no known, general algorithm that can solve 3-SAT in polynomial time. But, it has also not been proven that such an algorithm doesn't exist. It seems that the best known lower bounds for SAT sit somewhere in the polynomial range of $n^{1.801}$, though this has some other caveats about ``time-space tradeoffs'' which I don't fully understand \cite{SE3satlowerbounds}. Proving that SAT, for example, had an exponential (or even super polynomial) lower bound would, of course, establish that $P \neq NP$, since SAT is NP-complete, and this would serve to separate $P$ from $NP$. Of course, one could also prove $P=NP$ by simply giving a polynomial time algorithm for SAT i.e. by dropping the upper bound from exponential to polynomial. This might be ``easier'', in the sense that you would only have to find a single algorithm, but it may be ``harder" in the sense that P=NP may not actually be true!

\subsection{NP vs. NEXP}

I believe that one straightforward way to characterize the difference between ``polynomial" non-deterministic Turing machines (NP) and ``exponential" non-deterministic Turing machines (NEXP) is to consider the maximum depth of the computation tree branches. In NP, we are allowed to use non-deterministic machines with paths of polynomial length, whereas in NEXP, we are allowed to use non-deterministic machines with exponentially long paths.

Note that we can also use the nondeterministic Turing machine perspective as an alternate way to view the difference between classes like $NP$ and $coNP$. $NP$ consists of problems that have some accepting path out of all branches on the nondeterministic computation, whereas problems in $coNP$ are those for which all paths must accept. For example, satisfiability vs. unsatisfiability.


% Proving 
% That is, there is no lower bound on 3-SAT that is 
%  lower bound on 3-SAT is not polynomial. If this were proven, it would resolve the P=NP question i.e. it would establish that SAT, which is NP-complete, is indeed not in P.

% \begin{center}
%     \begin{tikzpicture}
%         % \node[circle] at (0,0);
%         \node[align=center, color=darkgray] at (0,1.8) {Outer};
%     \end{tikzpicture}
% \end{center}

% [1] https://cstheory.stackexchange.com/questions/1060/best-upper-bounds-on-sat#:~:text=9%20Answers&text=The%20best%20algorithm%20for%203,have%20not%20been%20analyzed%20yet.

\bibliographystyle{plain}
\bibliography{../../references.bib}

\end{document}