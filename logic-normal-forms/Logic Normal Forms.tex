\documentclass[10pt,a4paper]{article}
\usepackage[utf8]{inputenc}
\usepackage{amsmath}
\usepackage{amsfonts}
\usepackage{amssymb}
\usepackage{geometry}
\usepackage{verbatim}
\usepackage{enumerate}
\usepackage{fancyvrb}
\usepackage{graphicx}
\usepackage{tikz}
\usetikzlibrary{positioning}
\usetikzlibrary{shapes,snakes}
\usepackage[english]{babel}

\geometry{legalpaper, margin=1.5in}

\author{William Schultz}
\begin{document}
\title{Logic Normal Forms}
\author{William Schultz}
\maketitle

\begin{itemize}
    \item \textbf{Conjunctive normal form (CNF)}: a conjunction of disjunctions. More precisely, a conjunction of \textit{clauses}, where a clause is a disjunction of literals e.g.
    \begin{align*}
        (a \vee b) \wedge (b \vee \neg c) \wedge (a \vee d)
    \end{align*}
    Also called \textit{product of sums} form.
    \item \textbf{Disjunctive normal form (DNF)}: a disjunction of conjunctions. Also described as a disjunction of \textit{cubes}, where a cube is a conjunction of literals. e.g.
    \begin{align*}
        (a \wedge b) \vee (b \wedge \neg c) \vee (a \wedge d)
    \end{align*}
    Also called \textit{sum of products} form.
\end{itemize}


\end{document}