\documentclass[10pt]{article}

\usepackage{amsmath}
\usepackage{amssymb}
\usepackage{hyperref}
\usepackage{todonotes}
% \usepackage{biblatex}
\usepackage[margin=1.5in]{geometry}
\usepackage{tikz}

\hypersetup{pdftex,colorlinks=true,allcolors=blue}
\usepackage{hypcap}

\begin{document}

\title{Abstract Interpretation}
\author{William Schultz}
\date{\today}

\maketitle


If we want to analyze the behavior of a program, we typically perform some kind of \textit{abstraction}. That is, we approximate the concrete semantics of the program in some way that is sufficient for analysis. \textit{Abstract interpretation} provides a formal framework for defining and performing these types of program abstractions.

\subsection*{Background}

A \textit{lattice} $L=(S,\sqsubseteq)$ is a partially ordered set where each pair of elements has a least upper bound (i.e. \textit{join} $\sqcup$) and a greatest lower bound (i.e. \textit{meet}, $\sqcap$).

\subsection*{Abstraction Domains}

An \textit{abstraction domain} is defined as follows, where $D$ is a \textit{concrete domain} and $\hat{D}$ is an \textit{abstract domain}, and elements of $\hat{D}$ form a lattice:
\begin{itemize}
    \item $\gamma : \hat{D} \rightarrow D$: \textbf{Concretization function} that maps abstract values to sets of concrete elements.
    \item $\alpha : D \rightarrow \hat{D}$: \textbf{Abstraction function} that maps sets of concrete elements to the most precise value in the abstract domain. 
\end{itemize}
where $\alpha$ and $\gamma$ must form a \textit{Galois connection}. That is, they satisfy the following condition
\begin{align*}
    \forall x \in D, \forall \hat{x} \in \hat{D} : \alpha(x) \sqsubseteq \hat{x} \Leftrightarrow x \sqsubseteq \gamma(\hat{x}) 
\end{align*} 
Intuitively, this means that the abstraction and concretization functions respect the orderings of $D$ and $\hat{D}$. That is, if $\alpha(x)$, the abstraction of $x$, is ordered before some other $\hat{x} \in \hat{D}$, then $x$ should be ordered before $\gamma(\hat{x})$, the concretization of $\hat{x}$.

Note that we also have \textit{abstract transformers}, which are functions $T: \hat{D} \rightarrow \hat{D}$, and which are \textit{sound} if they respect the over-approximation property i.e., if an original abstract element $d$ is an overapproximation of its concretization, then after applying the transformer to $d$, it will remain an overapproximation of the concretization of the transformed value. TODO: Make this clearer. 


\end{document}