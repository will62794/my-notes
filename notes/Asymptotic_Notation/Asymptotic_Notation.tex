\documentclass[10pt,a4paper]{article}
\usepackage[utf8]{inputenc}
\usepackage{amsmath}
\usepackage{amsfonts}
\usepackage{amssymb}
\usepackage{geometry}
\usepackage{verbatim}
\usepackage{enumerate}
\usepackage{fancyvrb}
\usepackage{graphicx}
\usepackage{tikz}
\usetikzlibrary{positioning}
\usetikzlibrary{shapes,snakes}
\usepackage[english]{babel}

\geometry{legalpaper, margin=1.5in}

\author{William Schultz}
\begin{document}
\title{Asymptotic Notation}
\author{William Schultz}
\maketitle
 
If we want to describe the runtime of a Turing machine (i.e. a program), we describe it based on the size of the input. That is, how does the runtime scale as the input size scales. If we were to look at the performance on a single input, it would be difficult to compare one Turing machine against another since its specific runtime could be affected by many other factors e.g. speed of the machine running the program, etc. So, we look at the \textit{asymptotic} behavior of the machine as input sizes go to infinity. There are a variety of notations for describing the worst case runtime, best case runtime, etc. The notations below describes the asymptotic behavior of a function $f(n)$ in terms of a function $g(n)$. In the Turing machine case, $f$ represents the runtime (i.e. the number of steps taken) of a machine as a function of the input size, $n$.

Note that for the classes below, like $O(g(n))$, the $\in$ and $=$ symbols are, in practice, often used interchangeably, even though the former is technically correct. For example, $O(g(n))$ describes an entire family of functions (i.e. all functions asymptotically bounded above by $g$), so saying $f = O(g(n))$ is really an abuse of notation. Writing $f \in O(g(n))$ is more precise.

\begin{itemize}
    \item \textbf{Big O (Upper Bound)}: $f(n) \in O(g(n))$
    
    Establishes an upper bound i.e. worst case complexity. Formally, $\exists c > 0$ such that as $n$ approaches $\infty$, $f(n) \leq c*g(n)$. That is, $f$ is bounded above by $g$, within some constant factor, as $n$ approaches infinity.

    \item \textbf{Big Omega (Lower Bound)}: $f(n) \in \Omega(g(n))$
    
    Establishes a lower bound i.e. best case complexity. More formally, $\exists c > 0$ such that as $n$ approaches $\infty$, $f(n) \geq c*g(n)$. That is, $f$ is bounded below by $g$, within some constant factor, as $n$ approaches infinity.

    \item \textbf{Big Theta}: $f(n) \in \Theta(g(n))$

    $f$ is bounded asymptotically both above and below by $g(n)$. That is, $f \in O(g(n))$ and $f \in \Omega(g(n))$.

    \item \textbf{Small O}: $f(n) \in o(g(n))$
    
    $f$ is dominated by $g$ asymptotically. Can think of this as a variant of Big O notation but saying something stronger i.e. $f \in o(g(n))$ implies $f \in O(g(n))$. The function $f$ is not only bounded above by $g$ but is dominated by $g$. In other words, $f$'s asymptotic growth is strictly less than $g$'s growth. Formally, we can state this by saying that $\forall c > 0$, as $n$ goes to infinity, $f(n) \leq c*g(n)$. As a concrete example, $x^2 \in o(x^3)$.

    \item \textbf{Small Omega}: $f(n) \in \omega(g(n))$
    
    $f$ dominates $g$ asymptotically. Similar to $o(g(n))$ but establishes a strict lower bound. It is a stronger statement than $f \in \Omega(g(n))$, saying that $f$'s asymptotic growth is strictly greater than $g$'s.
\end{itemize}

\end{document}