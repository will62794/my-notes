\documentclass[10pt,a4paper]{article}
\usepackage[utf8]{inputenc}
\usepackage{amsmath}
\usepackage{amsfonts}
\usepackage{amssymb}
\usepackage{geometry}
\usepackage{verbatim}
\usepackage{enumerate}
\usepackage{fancyvrb}
\usepackage{graphicx}
\usepackage{tikz}
\usetikzlibrary{positioning}
\usetikzlibrary{shapes,snakes}
\usepackage[english]{babel}

\geometry{legalpaper, margin=1.5in}

\author{William Schultz}
\begin{document}
\title{DSA Problems}
\author{William Schultz}
\maketitle

\newcommand{\concept}[1]{\textcolor{blue}{\textit{\textbf{#1}}}}

\subsection*{Merge k Sorted Lists}

\begin{itemize}
    \item \textbf{Problem}: Given a set of $k$ linked lists, each which are individually sorted in ascending order, merge all $k$ lists into one sorted list. 
    \item \textbf{Solution Idea}: The basic approach is to essentially just perform the \textit{merge} step of merge-sort. That is, if we are given a set of already sorted lists, we can merge them all into one sorted lists by repeatedly popping the smallest element from the remaining, non-empty lists and appending it to the output list.
    \item \textbf{Key Concepts}: 
    \begin{itemize}
        \item \concept{Mergesort Merging}
        \item \concept{Linked List Manipulation}
    \end{itemize}
    The essence of the solution is very straightforward as long as you know and understand the ideas behind mergesort i.e. knowing the core idea that you can merge a set of already sorted lists by incrementally choosing the smallest element from each.
\end{itemize}

\subsection*{Remove duplicates from sorted linked list}
\begin{itemize}
\item \textbf{Problem}: Given a sorted linked list, remove any duplicates from the list.
\item \textbf{Solution Idea}: Iterate through the linked list, but at each node look ahead to see how many nodes in front of you contain an identical value to your own. Update your current ``next" pointer to point to the first node after this block of identical nodes in front of you. Since the list is sorted, you know that any duplicates of the current value must be directly in front of you.
\item \textbf{Key Concepts}: 
\begin{itemize}
    \item \concept{Linked List Traversal}
    \item \concept{Linked List Deletion}
    \item \concept{Duplicate Detection by Sorting}
\end{itemize}
The underlying insight in the solution is to recognize that sorting a list can be used an easy mechanism for detecting duplicates. That is, in a sorted list, all duplicates of a particular item will always appear in contiguous ``blocks". Once you recognize this fact, then implementing the solution mostly requires a standard application of linked list iteration and linked list item deletion. Namely, that to delete an item $n_2$ from a linked list that appears in a list as $n1 \rightarrow n_2 \rightarrow n_3$, you simply update  the ``next" pointer of $n_1$ to point to $n_3$ instead of $n_2$. Recall that a basic linked list node is a $LinkedListNode(val, next)$ structure, where $val$ is the value of that node, and $next$ is a pointer to the next item in the list.
\end{itemize}

\subsection*{Intersection of two linked lists}
\begin{itemize}
\item \textbf{Problem}: Given two singly linked lists, return the node at which the two lists intersect. If they have no intersection, then return \textit{null}.
\item \textbf{Solution Idea}: This is similar to a \textit{lowest common ancestor} problem. One approach is to walk backwards to the root from one of the lists and keep track of all nodes seen along the way. Then, walk backwards from the other list and check for the first node you hit that was already seen, and that node should be the intersection point. Note that this uses $O(n)$ space, if $n$ the upper bound on the size of the linked lists. 

It's also possible to do it without using any extra space by using a cleverer 2 pointer approach with a bit of counting. If we walk back to the root in both lists we can record the longer of the two. Then, from this we know the difference in length between the two lists, $diff$. So, we can walk backwards by $diff$ pointers in the longer list, and then walk forwards from there in both lists at the same time, until we hit a point where both pointers are pointing to the same node.
\item \textbf{Key Concepts}:
\begin{itemize}
    \item \concept{Linked List Traversal}
    \item \concept{Lowest Common Ancestor (?)}
\end{itemize}
\end{itemize}

\subsection*{Reverse linked list}
\begin{itemize}
\item \textbf{Problem}: Given a singly linked list, reverse the list.
\item \textbf{Solution Idea}: Iterate over the list and at each node, re-arrange the \textit{next} pointer so it now points to the previous node rather than the next node.
\begin{align*}
    None \rightarrow a \rightarrow b \rightarrow c
\end{align*}
If $curr=a$ and $curr.next = b$, then to do the reversal we want to end up with $a.next = None$ and then step forward, ending up with $curr=b$. So, at each step of the traversal, we keep track of hte previous item we looked at, so that we can reverse the pointer of the current node to point to it. We also need to save a reference to the next node before we update it.
\item \textbf{Key Concepts}:
\begin{itemize}
    \item \concept{Linked List Traversal}
    \item \concept{Pointer Swapping (?)}
\end{itemize}
Need to have a solid grasp of how to traverse a linked list, but also need to have good confidence in how to update points in a few steps (similar to how we swap variables), without overwriting the info we need to continue.
\end{itemize}

\subsection*{Add two binary strings}
% \begin{itemize}
% \item \textbf{Problem}: Given a singly linked list, reverse the list.
% \item \textbf{Solution Idea}: Iterate over the list and at each node, re-arrange the \textit{next} pointer so it now points to the previous node rather than the next node.
% \begin{align*}
%     None \rightarrow a \rightarrow b \rightarrow c
% \end{align*}
% If $curr=a$ and $curr.next = b$, then to do the reversal we want to end up with $a.next = None$ and then step forward, ending up with $curr=b$. So, at each step of the traversal, we keep track of hte previous item we looked at, so that we can reverse the pointer of the current node to point to it. We also need to save a reference to the next node before we update it.
% \item \textbf{Key Concepts}:
% \begin{itemize}
%     \item \concept{Linked List Traversal}
%     \item \concept{Pointer Swapping (?)}
% \end{itemize}
% Need to have a solid grasp of how to traverse a linked list, but also need to have good confidence in how to update points in a few steps (similar to how we swap variables), without overwriting the info we need to continue.
% \end{itemize}

\subsection*{Subsets of a list}

% \bibliographystyle{plain}
% \bibliography{../../references.bib}

\end{document}