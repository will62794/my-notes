\documentclass[10pt,a4paper]{article}
\usepackage[utf8]{inputenc}
\usepackage{amsmath}
\usepackage{amsfonts}
\usepackage{amssymb}
\usepackage{geometry}
\usepackage{verbatim}
\usepackage{enumerate}
\usepackage{fancyvrb}
\usepackage{graphicx}
\usepackage{tikz}
\usetikzlibrary{positioning}
\usetikzlibrary{shapes,snakes}
\usepackage[english]{babel}

\geometry{legalpaper, margin=1.5in}

\author{William Schultz}
\begin{document}
\title{DS \& Algorithms Problems}
\author{William Schultz}
\maketitle

\subsection*{Merge k Sorted Lists}

\begin{itemize}
    \item \textbf{Problem}: Given a set of $k$ linked lists, each which are individually sorted in ascending order, merge all $k$ lists into one sorted list. 
    \item \textbf{Solution Idea}: The basic approach is to essentially just perform the \textit{merge} step of merge-sort. That is, if we are given a set of already sorted lists, we can merge them all into one sorted lists by repeatedly popping the smallest element from the remaining, non-empty lists and appending it to the output list.
    \item \textbf{Takeaways}: The essence of the solution is very straightforward as long as you know and understand the ideas behind mergesort i.e. knowing the core idea that you can merge a set of already sorted lists by incrementally choosing the smallest element from each.
\end{itemize}


% \bibliographystyle{plain}
% \bibliography{../../references.bib}

\end{document}