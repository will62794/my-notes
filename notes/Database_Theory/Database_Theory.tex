\documentclass[10pt,a4paper]{article}
\usepackage[utf8]{inputenc}
\usepackage{amsmath}
\usepackage{amsfonts}
\usepackage{amssymb}
\usepackage{geometry}
\usepackage{verbatim}
\usepackage{enumerate}
\usepackage{fancyvrb}
\usepackage{graphicx}
\usepackage{tikz}
\usetikzlibrary{positioning}
\usetikzlibrary{shapes,snakes}
\usepackage[english]{babel}

\geometry{legalpaper, margin=1.5in}

\author{William Schultz}
\begin{document}
\title{Database Theory}
\author{William Schultz}
\maketitle

\section*{Joins}

At a high level, database (i.e. SQL) tables can be viewed as $n$-ary relations. Or, more plainly, as ``spreadsheets". For example, consider the following relation $A$
\begin{center}
    \begin{tabular}{ c c c }
     \textbf{Name} & \textbf{Age} & \textbf{Id} \\ 
     cell4 & cell5 & cell6 \\  
     cell7 & cell8 & cell9    
\end{tabular}
\end{center}




\end{document}