\documentclass[10pt,a4paper]{article}
\usepackage[utf8]{inputenc}
\usepackage{amsmath}
\usepackage{amsfonts}
\usepackage{amssymb}
\usepackage{geometry}
\usepackage{verbatim}
\usepackage{enumerate}
\usepackage{fancyvrb}
\usepackage{graphicx}
\usepackage{tikz}
\usetikzlibrary{positioning}
\usetikzlibrary{shapes,snakes}
\usepackage[english]{babel}

\geometry{legalpaper, margin=1.5in}

\author{William Schultz}
\begin{document}
\title{Database Theory}
\author{William Schultz}
\maketitle

\section*{Joins}

At a high level, database (i.e. SQL) tables can be viewed as $n$-ary relations (or, more plainly, as ``spreadsheets"). For example, consider the following relation $P$ (representing a set of homeowners)
\begin{center}
\begin{tabular}{| l | l | l |}
     \textbf{Name} & \textbf{Age} & \textbf{HouseId} \\ 
     Alice & 31 & 6 \\  
     Bob & 32 & 3 \\    
     Jane & 25 & 4    
\end{tabular}
\end{center}
and another relation $H$ (representing a set of houses)
\begin{center}
    \begin{tabular}{| l | l |}
         \textbf{Id} & \textbf{Year} \\ 
         6 & 1904 \\ 
         4 & 1965 
\end{tabular}
\end{center}
We can consider the \textit{cross product} of these two relations $P \times H$, which is simply the Cartesian product of all rows (i.e. tuples) in $P$ with all rows in $H$, giving relation $P \times H$ as
\begin{center}
    \begin{tabular}{| l | l | l | l | l |}
        \textbf{Name} & \textbf{Age}& \textbf{HouseId} & \textbf{Id} & \textbf{Year} \\
        Alice & 31 & 6 & 6 & 1904\\  
        Alice & 31 & 6 & 4 & 1965 \\  
        Bob & 32 & 3 & 6 & 1904 \\    
        Bob & 32 & 3 & 4 & 1965  \\    
        Jane & 25 & 4 & 6 & 1904  \\
        Jane & 25 & 4 & 4 & 1965   
\end{tabular}
\end{center}
with a total tuple count of $|P \times H| = |P| \times |H| = 6$.

On its own, the full cross product of two tables may not be very useful, but a commonly useful operation to apply after doing this cross product is the \textit{join}, which essentially just applies some filter (e.g. predicate) to the tuples that are generated as a result of this cross product operation. If we filter the result $P \times H$ based on the predicate $HouseId == Id$, then we say we're ``joining'' the two tables, $P$ and $H$, on $HouseId == Id$, which gives as a result:
\begin{center}
    \begin{tabular}{| l | l | l | l | l |}
        \textbf{Name} & \textbf{Age}& \textbf{HouseId} & \textbf{Id} & \textbf{Year} \\
        Alice & 31 & 6 & 6 & 1904\\  
        Jane & 25 & 4 & 4 & 1965   
\end{tabular}
\end{center}
which is basically the set of people in $P$ associated with the house in $H$ they own. More compactly, we typically notate a join on predicate $p$ between two relations $A$ and $B$ as
\begin{align*}
    A \Join_p B
\end{align*}
Again, we can think of this as simply a composition of cross product and filtering operations i.e.
\begin{align*}
    A \Join_p B = \sigma_p(A \times B)
\end{align*}
where $\sigma_p$ represents the filtering operation for a given predicate $p$ on tuples.

Note that joins are \textit{commutative} i.e. $A \Join_p B = B \Join_p A$. This can be easily seen from examining the decomposed form of joins in terms of cross products and filtering i.e. since $A \times B = B \times A$ (if we ignore ordering of columns in the output tuples). Note also that we can view a sequence of joins as a big cross product followed by a filtering operation at the end i.e.
\begin{align*}
    (A \Join_p B) \Join_q C = \sigma_q(\sigma_p(A \times B) \times C) = \sigma_{p \wedge q}(A \times B \times C)
\end{align*}
Furthermore, joins are \textit{associative}. This means that we can re-order joins as we please (since they are both \textit{associative} and \textit{commutative}), so there may be many different join executing orderings that produce the same final result (TODO: query optimization). That is,
\begin{align*}
    (A \Join_p B) \Join_q C \\ 
    A \Join_p (B \Join_q C) \\
    A \Join_p (C \Join_q B) \\
    (A \Join_p C) \Join_q B \\
    (C \Join_p A) \Join_q B
\end{align*}
are all equivalent. Some orderings may, however, be much more efficient to execute. More generally, we can think of a particular join ordering as a binary tree, that essentially maps to the syntactic parse tree of the above expressions. 

\end{document}