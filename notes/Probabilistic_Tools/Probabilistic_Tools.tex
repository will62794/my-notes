\documentclass[10pt,a4paper]{article}
\usepackage[utf8]{inputenc}
\usepackage{amsmath}
\usepackage{amsfonts}
\usepackage{amssymb}
\usepackage{geometry}
\usepackage{verbatim}
\usepackage{enumerate}
\usepackage{fancyvrb}
\usepackage{graphicx}
\usepackage{hyperref}
\usepackage{tikz}
\usetikzlibrary{positioning}
\usetikzlibrary{shapes,snakes}
\usepackage[english]{babel}

\geometry{legalpaper, margin=1.5in}

\author{William Schultz}
\begin{document}
\title{Probabilistic Tools}
\author{William Schultz}
\maketitle

\newcommand{\expect}[1]{\mathbb{E}\left[#1\right]}
\newcommand{\var}[1]{\text{Var}(#1)}

See \cite{Doerr_2019} for a reference on some probabilistic tools.

\subsection*{Union Bound}

Let $E_1,\dots,E_n$ be arbitrary events in some probability space. Then,
\begin{align*}
    Pr \left[ \bigcup_{i=1}^n E_i \right] \leq \sum_{i=1}^n Pr[E_i]
\end{align*}
That is, for a set of events $E_1,\dots,E_n$, the probability of at least one event occurring is less than or equal to the sum of the probabilities of the individual events.

\subsection*{Expectation and Variance}

For a discrete random variable $X$ taking values in some set $\Omega \subseteq \mathbb{R}$, its \textit{expectation} is defined as
\begin{align*}
    \expect{X} = \sum_{\omega \in \Omega} \omega \cdot Pr\left[X=\omega\right]
\end{align*}
and its \textit{variance} is defined as 
\begin{align*}
    \var{X} = \expect{(X - \expect{X})^2}
\end{align*}
That is, expectation is essentially a weighted sum of the values that the random variable can take on, where each value is weighted by the probability of that event occurring, and variance is essentially a measure of the average deviation of the variable from its expectation/mean.

\subsection*{Markov's Inequality}

\textit{Markov's inequality} is an elementary large deviation bound valid for \textit{all} non-negative random variables. Let $X$ be a non-negative random variable with $\expect{X} > 0$. Then, for all $\lambda > 0$
\begin{align*}
    Pr(X \geq \lambda \expect{X}) \leq \dfrac{1}{\lambda}
\end{align*}
That is, this establishes a bound, for some given parameter $\lambda$, on how likely a random variable is to be far away from its mean. For example, if $\lambda = 10$, then this means that the probability that $X$ is greater than 10 times its mean is $\leq \frac{1}{10}$. Note that this bound doesn't take into account anything about the actual distribution (only about its mean), so it may serve as only a rough estimate.

\subsection*{Chebyshev's Inequality}

Let $X$ be a random variable with $\var{X} > 0$, and where $\sigma = \sqrt{\var{X}}$ is its standard deviation. Then, for all $\lambda > 0$
\begin{align*}
    Pr \left[ \left| X - \expect{X} \right| \geq \lambda \sigma \right] \leq \dfrac{1}{\lambda^2}
\end{align*}
This bound tells us something about how far a random variable is from its expectation, in terms of the variance of the RV. That is, it puts a bound on the probability of how many ($\lambda$) standard deviations ($\sigma$) away from its mean $X$ may be.

% Note that Chebyshev's inequality can be derived from Markov's inequality. Start with
% \begin{align*}
%     Pr(X \geq \lambda \expect{X}) \leq \dfrac{1}{\lambda}
% \end{align*}

\subsection*{Chernoff Bound}

TODO.

\bibliographystyle{plain}
\bibliography{../../references.bib}

\end{document}