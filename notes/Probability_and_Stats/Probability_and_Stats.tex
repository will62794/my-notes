\documentclass[10pt,a4paper]{article}
\usepackage[utf8]{inputenc}
\usepackage{amsmath}
\usepackage{amsfonts}
\usepackage{amssymb}
\usepackage{geometry}
\usepackage{verbatim}
\usepackage{enumerate}
\usepackage{fancyvrb}
\usepackage{graphicx}
\usepackage{tikz}
\usepackage{upgreek}
\usepackage{hyperref}

\usetikzlibrary{positioning}
\usetikzlibrary{shapes,snakes}
\usepackage[english]{babel}

\geometry{legalpaper, margin=1.5in}

\author{William Schultz}
\begin{document}
\title{Probability and Stats}
\author{William Schultz}
\maketitle


\section{Random Variables}

We can model probabilistic events based on \textit{outcomes}, which are the outcome of an experiment or a trial (e.g., getting a ``3'' on the roll of a die), and a \textit{sample space}, which is a set of possible outcomes, and \textit{events}, which are a subset of the sample space. 

A \textbf{random variable} is then, formally, simply a mapping from outcomes in the sample space to the set of real numbers. For example consider a 6-sided die with sides $\{1,2,3,4,5,6\}$. Some possible outcomes are $1$, $3$, or $5$, and the sample space is $\{1,2,3,4,5,6\}$. The probability of each outcome for a fair die, deteremined by some random variable, is $1/6$. We can define an event $A$ representing the case that the roll is odd i.e., $A=\{1,3,5\}$.

\subsection{Expected Value}
The \textbf{expected value} of a random variable $X$ can be thought of as the ``average'' or ``mean'' value attained by that random variable. Formally, expected value is defined as
\begin{align*}
    E[X] = \sum_{x} x * P(X=x)
\end{align*}
for every outcome $x$ in the sample space of $X$. For example, for a fair, 6-sided die, the expected value is 
\begin{align*}
  \frac{1}{6}*1 + \frac{1}{6}*2 + \frac{1}{6}*3 + \frac{1}{6}*4 + \frac{1}{6}*5 + \frac{1}{6}*6 = 3.5
\end{align*}
Or if we have, say, a 4-sided die where each roll has some ``payoff'', e.g. 
\begin{itemize}
    \item 1 $\mapsto$ 1
    \item 2 $\mapsto$ 2
    \item 3 $\mapsto$ 3
    \item 4 $\mapsto$ 10
\end{itemize}
our expected payoff from rolling this die is
\begin{align*}
    \frac{1}{4}*1 + \frac{1}{4}*2 + \frac{1}{4}*3 + \frac{1}{4}*10 = 4
\end{align*}

\section*{Some Examples}

\paragraph{Coin Toss Game} Assume there are two gamblers playing a coin toss game. Gambler $A$ has $(n+1)$ fair coins, and $B$ has $n$ fair coins. What is the probability that $A$ will have more heads than $B$ if both flip all of their coins?

We cna look at this problem symmetrically if we imagine a slightly simplified scenario where $A$ and $B$ both have $n$ coins. Then, we know that for the following 3 scenarios
\begin{itemize}
    \item $A$ flips more heads than $B$.
    \item $A$ flips the same number of heads as $B$.
    \item $A$ flips fewer heads than $B$. 
\end{itemize}
the first and third are symmetric, and so must have equal probabilities. Then, we really only need to consider the second case, which is the case where $A$'s $n+1$-th flip would actually make a difference.

\paragraph*{Card Game} A casino offers a card game from a deck of 52 cards with 4 cards each for $2,3,4,5,6,7,8,9,10,J,Q,K,A$. You pick up a card from the deck and the dealer picks another one with replacement. What is the probability that you picked a larger card than the dealer's?

Well, one way to analyze this is by considering all possible choices of two cards $C_1$ and $C_2$. The probabibility of pciking any particular suit for $C_1$ is $4/52 = 1/13$. We can choose to then simply analyze the probability of getting a smaller card for each possible suit case. Basically, for each suit, the probability of picking a lesser card for $C_2$ is $4/51 * k$, where $k$ is 1 less than the rank of the suit. So, in general, we can compute the overall probability as
\begin{align*}
    \left(\frac{1}{13}*\frac{4}{51}*0\right) + \left(\frac{1}{13}*\frac{4}{51}*1\right) + \left(\frac{1}{13}*\frac{4}{51}*2\right) + \dots + \left(\frac{1}{13}*\frac{4}{51}*12\right) = 0.4705
\end{align*}


% \bibliographystyle{alpha}
% \bibliography{../../references.bib}

\end{document}