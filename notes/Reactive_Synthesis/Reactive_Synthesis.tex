\documentclass[10pt,a4paper]{article}
\usepackage[utf8]{inputenc}
\usepackage{amsmath}
\usepackage{amsfonts}
\usepackage{amssymb}
\usepackage{geometry}
\usepackage{verbatim}
\usepackage{enumerate}
\usepackage{fancyvrb}
\usepackage{graphicx}
\usepackage{tikz}
\usepackage{hyperref}
\usetikzlibrary{positioning}
\usetikzlibrary{shapes,snakes}
\usepackage[english]{babel}

\geometry{legalpaper, margin=1.5in}

\author{William Schultz}
\begin{document}
\title{Reactive Synthesis}
\author{William Schultz}
\maketitle

% \section{Reactive synthesis}

\begin{itemize}
    \item The \textbf{verification problem} is: given system $M$ and spec/property $\varphi$, check that $M \vDash \varphi$.
    \item The \textbf{synthesis problem} is: given spec $\varphi$ synthesize $M$ such that $M \vDash \varphi$.
\end{itemize}
The \textit{deductive approach} \cite{manna1980deductive} tries to synthesize an input/output program by extracting it from a realizability proof.

\textit{Temporal synthesis} considers specifications given in the form of LTL, for example. Initial approach was to use satisfiability of a temporal formula as a way to derive $M$ \cite{1981clarkemerson}.


\bibliographystyle{plain}
\bibliography{../../references.bib}

\end{document}